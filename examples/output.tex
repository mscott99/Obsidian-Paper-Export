\documentclass{article}
\usepackage{hyperref}
\usepackage{amsmath}
\usepackage{amsthm}
\usepackage{amssymb}
\newtheorem{theorem}{Theorem}
\newtheorem{lemma}[theorem]{Lemma}
\newtheorem{corollary}[theorem]{Corollary}
\newtheorem{definition}{Definition}
\title{Title}
\author{Author}
\begin{document}
\maketitle
\begin{abstract}
This is a simple example to showcase the Obsidian to LaTeX converter.

\end{abstract}

\section{Introduction}\label{section:Introduction}
This document demonstrates the conversion of Markdown notes to a LaTeX document, including internal links and embedded content.
\section{Results}\label{section:Results}
We present the following lemma:
\begin{lemma}
\label{lemma_1:Statement}
Every even integer greater than 2 can be expressed as the sum of two prime numbers.

\end{lemma}
The main theorem is:
\begin{theorem}
\label{theorem_1:Statement}
For every positive integer  $n$ , the sum of the first  $n$  odd integers is equal to  $n^2$ .

\end{theorem}

\section{Proofs}\label{section:Proofs}
Here is the proof for the main theorem. The proof is specifically for \autoref{theorem_1:Statement}.
\begin{proof}
We proceed by induction.
Base case ( $n = 1$ ): The sum of the first odd integer (1) is equal to  $1^2$ , which is true.
Inductive step: Assume that the sum of the first  $k$  odd integers is equal to  $k^2$ . We want to show that the sum of the first  $k+1$  odd integers is equal to  $(k+1)^2$ .
The sum of the first  $k$  odd integers is  $k^2$ . The next odd integer is  $(2k+1)$ . Therefore, the

\end{proof}

\end{document}
