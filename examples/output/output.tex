\documentclass{article}
\usepackage{bbm}
\usepackage{amssymb}
\usepackage{amsmath}
\usepackage{amsthm}
\usepackage{aliascnt}
\usepackage{mathtools}
\usepackage{biblatex}
\usepackage{graphicx}
\usepackage{mathrsfs}

%\usepackage[noadjust]{cite}


\usepackage{xr-hyper}

%\usepackage[colorlinks = true, linkcolor = linkcolor, urlcolor=linkcolor, citecolor = linkcolor, filecolor=linkcolor]{hyperref}
\usepackage{hyperref}
%\definecolor{linkcolor}{HTML}{005075}
%\definecolor{ForestGreen}{HTML}{009B55}

\theoremstyle{plain}
\newtheorem{theorem}{Theorem}[section]
% \newtheorem{prop}{Proposition}[section]
% \newtheorem{coro}{Corollary}
% \newtheorem{lem}{Lemma}

\renewcommand{\sectionautorefname}{Section} % name for \autoref
\renewcommand{\subsectionautorefname}{Section} % name for \autoref
\renewcommand{\subsubsectionautorefname}{Section} % name for \autoref

% prop
\newaliascnt{proposition}{theorem}% alias counter "<newTh>"
\newtheorem{proposition}[proposition]{Proposition}
\aliascntresetthe{proposition}
\providecommand*{\propositionautorefname}{Proposition} % name for \autoref

% coro
\newaliascnt{corollary}{theorem}% alias counter "<newTh>"
\newtheorem{corollary}[corollary]{Corollary}
\aliascntresetthe{corollary}
\providecommand*{\corollaryautorefname}{Corollary} % name for \autoref

% lem
\newaliascnt{lemma}{theorem}% alias counter "<newTh>"
\newtheorem{lemma}[lemma]{Lemma}
\aliascntresetthe{lemma}
\providecommand*{\lemmaautorefname}{Lemma} % name for \autoref


\theoremstyle{definition}
\newtheorem{definition}{Definition}[section]
\newtheorem{example}{Example}


\theoremstyle{remark}
\newtheorem{rmk}{Remark}[section]
\newtheorem{fact}[rmk]{Fact}
\newtheorem*{rmk*}{Remark}
\newtheorem*{fact*}{Fact}

\newenvironment{remark}
{\pushQED{\qed}\renewcommand{\qedsymbol}{$\diamond$}\rmk}
{\popQED\endrmk}

\providecommand*{\remarkautorefname}{Remark} % name for \autoref
\providecommand*{\rmkautorefname}{Remark} % name for \autoref
\providecommand*{\definitionautorefname}{Definition} % name for \autoref


\newcommand{\sphere}[1]{\mathbb{S}^{#1-1}}

% My new commands
\newcommand{\bP}{\mathbb{P}}
\newcommand{\bE}{\mathbb{E}}
\newcommand{\bR}{\mathbb{R}}
\newcommand{\bN}{\mathbb{N}}
\newcommand{\bB}{\mathbb{B}}

% \newcommand{\bI}{\mathds{1}}
\newcommand{\fro}[1]{\|#1\|_{F}}
\newcommand{\norm}[1]{\|#1\|}
\newcommand{\normbig}[1]{\left\|#1\right\|}

\newcommand{\argmin}{\mathop{\arg \min}}

\addbibresource{bibliography.bib}

\title{./examples/main_note.md}
\author{Author}

\begin{document}
\maketitle
\begin{abstract}
This is a simple example to showcase the Obsidian to LaTeX converter.


\end{abstract}

\section{Introduction}
\label{section:Introduction}
This document demonstrates the conversion of Markdown notes to a LaTeX document, including internal links and embedded content.

\section{Results}
\label{section:Results}

\begin{align*}
  \|\hat x - x_0\|_2
  \leq \|x^\perp\|_2 + 3\|\tilde{F}x^\perp\|_2 + 3 \|\eta\|_2 + \frac{3}{2}\hat\varepsilon.
\end{align*}
Here is some  $\sum a_i$ .

\begin{equation*}
\sum_{i=1}^k A_i
\end{equation*}

\begin{equation*}
\sum_{i=1}^n A_3
\end{equation*}

\begin{equation*}
\sum_{i=1}^n A_3
\end{equation*}
We present the following lemma:

\begin{lemma}[lemma_1]
\label{lemma_1}
Every even integer greater than 2 can be expressed as the sum of two prime numbers.


\end{lemma}

The main theorem is:

\begin{theorem}[theorem_1]
\label{theorem_1}
For every positive integer  $n$ , the sum of the first  $n$  odd integers is equal to  $n^2$ .


\end{theorem}


\section{Proofs}
\label{section:Proofs}
Here is the proof for the main theorem. The proof is specifically for \autoref{theorem_1}. I may or may not follow from \cite{rudelsonSparseReconstructionFourier2008}.

\begin{proof}[Proof of~{\autoref{theorem_1}}]
\label{proof:theorem_1}
We proceed by induction.

Base case ( $n = 1$ ): The sum of the first odd integer (1) is equal to  $1^2$ , which is true.

Inductive step: Assume that the sum of the first  $k$  odd integers is equal to  $k^2$ . We want to show that the sum of the first  $k+1$  odd integers is equal to  $(k+1)^2$ .

The sum of the first  $k$  odd integers is  $k^2$ . The next odd integer is  $(2k+1)$ . Therefore, the


\end{proof}


\printbibliography
\end{document}
