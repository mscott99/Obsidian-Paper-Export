\documentclass{article}
\usepackage{bbm}
\usepackage{amssymb}
\usepackage{amsmath}
\usepackage{amsthm}
\usepackage{aliascnt}
\usepackage{mathtools}
\usepackage{biblatex}
\usepackage{graphicx}
\usepackage{mathrsfs}

%\usepackage[noadjust]{cite}


\usepackage{xr-hyper}

%\usepackage[colorlinks = true, linkcolor = linkcolor, urlcolor=linkcolor, citecolor = linkcolor, filecolor=linkcolor]{hyperref}
\usepackage{hyperref}
%\definecolor{linkcolor}{HTML}{005075}
%\definecolor{ForestGreen}{HTML}{009B55}

\theoremstyle{plain}
\newtheorem{theorem}{Theorem}[section]
% \newtheorem{prop}{Proposition}[section]
% \newtheorem{coro}{Corollary}
% \newtheorem{lem}{Lemma}

\renewcommand{\sectionautorefname}{Section} % name for \autoref
\renewcommand{\subsectionautorefname}{Section} % name for \autoref
\renewcommand{\subsubsectionautorefname}{Section} % name for \autoref

% prop
\newaliascnt{proposition}{theorem}% alias counter "<newTh>"
\newtheorem{proposition}[proposition]{Proposition}
\aliascntresetthe{proposition}
\providecommand*{\propositionautorefname}{Proposition} % name for \autoref

% coro
\newaliascnt{corollary}{theorem}% alias counter "<newTh>"
\newtheorem{corollary}[corollary]{Corollary}
\aliascntresetthe{corollary}
\providecommand*{\corollaryautorefname}{Corollary} % name for \autoref

% lem
\newaliascnt{lemma}{theorem}% alias counter "<newTh>"
\newtheorem{lemma}[lemma]{Lemma}
\aliascntresetthe{lemma}
\providecommand*{\lemmaautorefname}{Lemma} % name for \autoref


\theoremstyle{definition}
\newtheorem{definition}{Definition}[section]
\newtheorem{example}{Example}


\theoremstyle{remark}
\newtheorem{rmk}{Remark}[section]
\newtheorem{fact}[rmk]{Fact}
\newtheorem*{rmk*}{Remark}
\newtheorem*{fact*}{Fact}

\newenvironment{remark}
{\pushQED{\qed}\renewcommand{\qedsymbol}{$\diamond$}\rmk}
{\popQED\endrmk}

\providecommand*{\remarkautorefname}{Remark} % name for \autoref
\providecommand*{\rmkautorefname}{Remark} % name for \autoref
\providecommand*{\definitionautorefname}{Definition} % name for \autoref


\newcommand{\sphere}[1]{\mathbb{S}^{#1-1}}

% My new commands
\newcommand{\bP}{\mathbb{P}}
\newcommand{\bE}{\mathbb{E}}
\newcommand{\bR}{\mathbb{R}}
\newcommand{\bN}{\mathbb{N}}
\newcommand{\bB}{\mathbb{B}}

% \newcommand{\bI}{\mathds{1}}
\newcommand{\fro}[1]{\|#1\|_{F}}
\newcommand{\norm}[1]{\|#1\|}
\newcommand{\normbig}[1]{\left\|#1\right\|}

\newcommand{\argmin}{\mathop{\arg \min}}

\addbibresource{bibliography.bib}

\title{../../Ik-Vault/Zettelkasten/Sub-Gaussian McDiarmid Inequality and Classification on the Sphere.md}
\author{Author}

\begin{document}
\maketitle
\begin{abstract}
We use the sub-Gaussian McDiarmid inequality to quantify the parametric error for binary classification on the sphere. We also include a proof of this inequality, which employs the entropy method.


\end{abstract}
#longform

\section{Introduction}
\label{section:Introduction}
#project/toplevel 
Consider a binary linear classification problem with feature vectors  $X_1, X_2, \dots, X_m \stackrel{\text{iid}}{\sim} \mathrm{Unif}(\mathbb{S}^{n-1})$  and corresponding labels  $Y_i = \mathrm{sign}(\langle w, X_i \rangle)$ , where  $w \in \mathbb{S}^{n-1}$  is fixed. The objective is to estimate  $w$  (or equivalently, learn the linear classifier  $x \mapsto \langle w, \cdot \rangle$ ). Here, we study the statistical properties of the following quantity

\begin{align}
\widetilde{w} := \frac{1}{m}\sum_{i=1}^m X_i Y_i,
\end{align}
which yields the estimator

\begin{align}
\widehat{w} := \frac{\tilde{w}}{\norm{\bE\tilde{w}}_{2}}.
\end{align}
The deviation

\begin{align*}
\norm{\widetilde{w} - \bE \widetilde{w}}_2 = \normbig{\frac{1}{m}\sum_{i=1}^m X_i Y_i - \bE \widetilde{w}}_2 = \normbig{\frac{1}{m}\sum_{i=1}^m Z_i - \bE Z_1}_2,
\end{align*}
where  $Z_i := X_i Y_i$  is now distributed uniformly on a half-sphere, is a well-behaved function of independent random variables. Hence, is amenable to concentration of measure principles. Here, we control the sub-Gaussian norm of this quantity using the  \textit{sub-Gaussian McDiarmid inequality}  \cite{maurerConcentrationInequalitiesSubGaussian2021}.


<<<<<<< Updated upstream
=======
\begin{theorem}[Sub-Gaussian McDiarmid]
\label{Sub-Gaussian McDiarmid}
Given  $f:\mathcal{X}^m \to \mathbb{R}$  and random variables  $\{ X_i \}_{i \in [m]}$ , suppose that  $\forall i \in [m]$ ,


\begin{equation*}
\lVert  f(X_1, \dots, X_m) - \mathbb{E}f(X_1, \dots, X_m)\rVert_{\psi_2|X^{(i)}} \leq K \quad a.s.
\end{equation*}

Then


\begin{equation*}
\lVert f(X_1, \dots, X_m) - \mathbb{E}f(X_1, \dots, X_m)\rVert_{\psi_2} \le K\sqrt{m}.
\end{equation*}


\end{theorem}


>>>>>>> Stashed changes
\begin{theorem}[Characterisation of Estimation Error]
\label{Characterisation of Estimation Error}
Suppose  $n \in \mathbb{N}$ . Then we have


\begin{equation*}
\mathbb{E}\left[\|\widehat{w}-w\|_2\right] \asymp \sqrt{\frac{n}{m}},
\end{equation*}

and


\begin{equation*}
\left\lVert \left\lVert \widehat{w} - w \right\rVert_2 - \mathbb{E}\left[\left\lVert \widehat{w} - w \right\rVert_2 \right] \right\rVert_{\psi_2} \lesssim  \frac{1}{\sqrt{ m }}.
\end{equation*}


\end{theorem}


\section{Proof of \autoref{Characterisation of Estimation Error}}
\label{section:Proof of \autoref{Characterisation of Estimation Error}}
<<<<<<< Updated upstream
proof::We shall use the notation introduced in \autoref{Sub-Gaussian McDiarmid Inequality and Classification on the Sphere}. Without loss of generality, we may assume  $w = e_1$ . We compute
=======
proof::
\begin{proof}[Proof of~{\autoref{Characterisation of Estimation Error}}]
\label{proof:Characterisation of Estimation Error}
We shall use the notation introduced in \autoref{Sub-Gaussian McDiarmid Inequality and Classification on the Sphere}. Without loss of generality, we may assume  $w = e_1$ . We compute
>>>>>>> Stashed changes


\begin{align*}
\|\mathbb{E} [\widetilde{w}]\|_2 &= \frac{\int_0^1 y (1-y^2)^{(n-3)/2}dy}{\int_0^1 (1-y^2)^{(n-3)/2} dy}\\
&= \frac{2 \Gamma(n/2)}{\sqrt{\pi} (n-1) \Gamma((n-1)/2)}\\
&\asymp \frac{1}{\sqrt{n}}. 
\end{align*}

We now compute the mean of the deviation  $\|\widetilde{w} -\bE[\widetilde{w}]\|_2$  as Further, we compute the deviation of  $\|\widetilde{w} -\bE[\widetilde{w}]\|_2$  from its mean as


\begin{equation*}
\| \widetilde{w} - \bE[\widetilde{w}]\|_2 = \normbig{\frac{1}{m}\sum_{i=1}^m Z_i - \bE[Z_1]}_2 =: g(Z_1, \dots, Z_m),
\end{equation*}

where  $Z_1, \dots, Z_m \stackrel{iid}{\sim} Unif(\sphere{n}_+)$ . Let  $i \in [m]$ . Let  $z_1, \dots, z_{i-1}, z_{i+1}, \dots, z_m \in \sphere{n}$  be fixed. Note that  $x \mapsto g(z_1, \dots, z_{i-1}, x, z_{i+1}, \dots, z_m)$   is a Lipschitz function from  $\sphere{n}$  to  $\bR$  with Lipschitz constant  $1/m$  since


\begin{equation*}
|g(z_1, \dots, z_{i-1}, x, z_{i+1}, \dots, z_m) - g(z_1, \dots, z_{i-1}, x', z_{i+1}, \dots, z_m)| \le \frac{1}{m}\left\|x-x'\right\|_2
\end{equation*}

<<<<<<< Updated upstream
by the reverse triangle inequality. Then it follows from \autoref{Concentration on the Half-Sphere} that
=======
by the reverse triangle inequality. Now consider the following lemma. 
\begin{lemma}[Concentration on the Half-Sphere]
\label{Concentration on the Half-Sphere}
For any lipschitz function  $f:\sphere{n}_+ \to \mathbb{R}$  with Lipschitz constant  $L$  and a random vector  $X$  distributed uniformly on  $\sphere{n}_+$ , we have


\begin{equation*}
\lVert f(X)-\mathbb{E}[f(X)]\rVert_{\psi_2} \lesssim \frac{L}{\sqrt{n}}.
\end{equation*}


\end{lemma}
  It follows that
>>>>>>> Stashed changes


\begin{equation*}
\normbig{g(z_1, \dots, z_{i-1}, Z_i, z_{i+1}, \dots, z_m) - \bE[g(z_1, \dots, z_{i-1}, Z_i, z_{i+1}, \dots, z_m)]}_{\psi_2} \lesssim \frac{1}{m\sqrt{n}}.
\end{equation*}

Finally, it follows from \autoref{Sub-Gaussian McDiarmid} that


\begin{equation*}
\|\| \widetilde{w} - \bE[\widetilde{w}]\|_2 - \bE[\| \widetilde{w} - \bE\left[\widetilde{w}]\|_2\right]\|_{\psi_2}= \|g(Z_1,\dots,Z_m) - \bE[g(Z_1,\dots,Z_m)]\|_{\psi_2} \lesssim \frac{1}{\sqrt{mn}}.
\end{equation*}

The final result now follows by combining the above estimate with the estimates on  $\|\bE[\widetilde{w}]\|_2$  and  $\|\widetilde{w} - \bE[\widetilde{w}]\|_2$  proved before.


<<<<<<< Updated upstream
proof::Uses \autoref{Concentration on the Half-Sphere} together with \autoref{Norm of Addition is Lipschitz}. From this we are done.



\section{Proof of \autoref{Sub-Gaussian McDiarmid}}
\label{section:Proof of \autoref{Sub-Gaussian McDiarmid}}
proof::Let  $Z := f(X_1, \dots, X_m) - \bE f(X_1, \dots, X_m)$ , and consider its log moment generating function  $\psi(\lambda) = \log \bE e^{\lambda Z}$ , for  $\lambda \in \bR$ . All we need to show is that  $\psi(\lambda) \leq \frac{mK^2\lambda^2}{2}$  so that by definition of sub-Gaussian random variables, we are done. In fact, it suffices to prove this for  $\lambda \geq 0$ , because the case  $\lambda < 0$  then follows by repeating the argument for  $-Z$ .
=======
\end{proof}
 
\begin{proof}[Proof of~{\autoref{Concentration on the Half-Sphere}}]
\label{proof:Concentration on the Half-Sphere}
Let  $\widetilde{X}$  be a random vector defined as


\begin{equation*}
\widetilde{X} = BX + (1-B) \gamma(X),
\end{equation*}

where  $B \sim Ber\left( \frac{1}{2} \right)$  and  $\gamma$  denotes reflection in the hyperplane  $x_1 = 0$ . Note that  $\widetilde{X}$  is distributed uniformly on  $\sphere{n}$ . We define  $\widetilde{f}: \sphere{n} \to \bR$  as


\begin{equation*}
\widetilde{f}(x) = f(\widetilde{\gamma}(x)),
\end{equation*}

where


\begin{equation*}
\widetilde{\gamma}(x) = \begin{cases}
x, & x \in \sphere{n}_+, \\
\gamma(x),  & \text{otherwise}.
\end{cases}
\end{equation*}

Note that  $\widetilde{f}$   is also  $L$ -Lipschitz since


\begin{equation*}
\|\widetilde{\gamma}(x_1) - \widetilde{\gamma}(x_2)\|_2 \le \|x_1 - x_2\|_2,
\end{equation*}

which implies


\begin{equation*}
\left|\widetilde{f}(x_1) - \widetilde{f}(x_2)\right| = \left|f(\widetilde{\gamma}(x_1)) - f(\widetilde{\gamma}(x_2))\right|  \le L \norm{\widetilde{\gamma}(x_1) - \widetilde{\gamma}(x_2)}_2 \le L\|x_1 - x_2\|_2.
\end{equation*}

Then it follows from concentration of Lipschitz functions on the sphere that


\begin{equation*}
\normbig{\widetilde{f}(\widetilde{X}) - \mathbb{E}\left[\widetilde{f}(\widetilde{X})\right]}_{\psi_2} \lesssim \frac{L}{\sqrt{n}}.
\end{equation*}

Finally, observing that  $\widetilde{f}(\widetilde{X})=f(X)$  yields the desired result.


\end{proof}


\section{Proof of \autoref{Sub-Gaussian McDiarmid}}
\label{section:Proof of \autoref{Sub-Gaussian McDiarmid}}
proof::
\begin{proof}[Proof of~{\autoref{Sub-Gaussian McDiarmid}}]
\label{proof:Sub-Gaussian McDiarmid}
Let  $Z := f(X_1, \dots, X_m) - \bE f(X_1, \dots, X_m)$ , and consider its log moment generating function  $\psi(\lambda) = \log \bE e^{\lambda Z}$ , for  $\lambda \in \bR$ . All we need to show is that  $\psi(\lambda) \leq \frac{mK^2\lambda^2}{2}$  so that by definition of sub-Gaussian random variables, we are done. In fact, it suffices to prove this for  $\lambda \geq 0$ , because the case  $\lambda < 0$  then follows by repeating the argument for  $-Z$ .
>>>>>>> Stashed changes

The proof we present here uses the  \textit{entropy method}  (see \cite{boucheronConcentrationInequalitiesNonasymptotic2016}, \cite{wainwrightHighDimensionalStatisticsNonAsymptotic2019}).  definition:: For a non-negative random variable  $W$ , and the convex function  $\phi(w) = w\log w$ , define the  $\phi$ -entropy of  $W$  as


\begin{equation*}
Ent(W) = \bE\,\phi(W) - \phi\left(\bE W\right)
\end{equation*}

 This quantity is well defined when both  
\begin{equation*}
W
\end{equation*}
  and  
\begin{equation*}
\phi(W)
\end{equation*}
  have finite expectations. Some basic properties of the entropy are provided in the appendix.

First, we bound the cumulant generating function by a function of the entropy.


\begin{lemma}[Herbst's Argument]
\label{Herbst's Argument}
To bound the cumulant generating function it suffices to bound the entropy. Specifically,


\begin{equation*}
\psi(\lambda) =\lambda \int_0^\lambda  \frac{Ent(e^{tZ})}{\varphi(t)t^2}  \, dt.
\end{equation*}


\end{lemma}
 Notice that if  
\begin{equation*}
Ent(e^{tZ}) \leq cmK^2 t^2 \varphi(t)
\end{equation*}
 , then it follows that


\begin{equation*}
\frac{\psi(\lambda)}{\lambda} \leq \int _0^\lambda cK^2 \frac{t^2}{t^2} \, dt = cK^2 \lambda^2
\end{equation*}


\begin{equation*}
\implies \psi(\lambda) \leq cmK^2 \lambda^2.
\end{equation*}

and from this the statement follows. Therefore we need only bound the entropy.

Let us first define some notation for the conditioning of random variables: let  $Z_i := \mathbb{E}[Z |X_1, \dots, X_i]$  and let  $Z^{(i)}:= \mathbb{E}[Z|X_1, \dots, X_{i-1}, X_{i+1}, \dots, X_m]$ .

<<<<<<< Updated upstream
We employ a standard method to bound the entropy: the so-called tensorization of entropy\ref{Tensorization of Entropy}. lemma::Define  $W = g(X_1, \dots, X_m)$ , then
=======
We employ a standard method to bound the entropy: the so-called tensorization of entropy\ref{Tensorization of Entropy}. lemma::
\begin{lemma}[Tensorization of Entropy]
\label{Tensorization of Entropy}
Define  $W = g(X_1, \dots, X_m)$ , then
>>>>>>> Stashed changes


\begin{equation*}
Ent(W) \leq \mathbb{E} \sum_{i=1}^m Ent^{(i)}(W)
\end{equation*}

where  $Ent^{(i)}(W) = \mathbb{E}^{(i)}[W \log W] - \mathbb{E}^{(i)}W \log \mathbb{E}^{(i)}W$ .

<<<<<<< Updated upstream
=======

\end{lemma}
>>>>>>> Stashed changes
 From this lemma we find that


\begin{equation*}
Ent(e^{-tZ}) \leq \sum_{i=1}^m \mathbb{E} Ent^{(i)}(e^{-tZ}).
\end{equation*}

We now bound  $Ent^{(i)}(e^{tZ})$ . Fix  $X_1, \dots, X_{i-1}, X_{i+1},\dots, X_m$ . Then notice that bounding  $Ent^{(i)}(e^{(-tZ)})$  reduces to bounding the entropy of a one-dimensional sub-gaussian random variable. The sub-Gaussianity in this setting is given by assumptions in the statement.

<<<<<<< Updated upstream
We bound this in the following lemma. lemma::Let  $Z$  be a sub-Gaussian random variable with norm  $K$ . Then
=======
We bound this in the following lemma. lemma::
\begin{lemma}[Bounding the Entropy of a Sub-Gaussian Random Variable]
\label{Bounding the Entropy of a Sub-Gaussian Random Variable}
Let  $Z$  be a sub-Gaussian random variable with norm  $K$ . Then
>>>>>>> Stashed changes


\begin{equation*}
Ent(e^{tZ}) \leq CK^2 t^2 \varphi(t).
\end{equation*}

<<<<<<< Updated upstream
=======

\end{lemma}
>>>>>>> Stashed changes
 We can now complete the proof. We write


\begin{align*}
\frac{Ent(e^{tZ})}{\varphi(t)} &\leq \frac{\mathbb{E}\sum_{i=1}^m Ent^{(i)}(e^{tZ})}{\varphi(t)}\\
& \leq \mathbb{E} \sum_{i=1}^m \frac{CK^2t^2\mathbb{E}^{(i)}e^{Zt}}{\varphi(t)} \\
& \leq CmK^2 t^2
\end{align*}

and as argued above, we then have


\begin{equation*}
\psi(\lambda) \leq c m K^2 \lambda^2
\end{equation*}

and therefore  $Z$  is sub-gaussian with norm  $K \sqrt{ m }$ .

<<<<<<< Updated upstream
 We now show the three lemmas that where used in the proof in order of appearance. proof::Observe that
=======

\end{proof}

We now show the three lemmas that where used in the proof in order of appearance.
proof::
\begin{proof}[Proof of~{\autoref{Herbst's Argument}}]
\label{proof:Herbst's Argument}
Observe that
>>>>>>> Stashed changes


\begin{equation*}
\frac{\psi(\lambda)}{\lambda} = \int_0^\lambda \frac{d}{dt} \frac{\psi(t)}{t} \, dt = \int_0^\lambda  \frac{Ent(e^{tZ})}{\varphi(t)t^2}  \, dt.
\end{equation*}

Indeed, this follows from computing the derivative of  $\frac{\psi(t)}{t}$ .


\begin{align*}
\frac{d}{dt} \frac{\psi(t)}{t} &= \frac{\psi'(t)t - \psi(t)}{t^2}\\
&= \frac{\left( \frac{\mathbb{E}[Ze^{tZ}]}{\varphi(t)}t - \frac{\varphi(t)\psi(t)}{\varphi(t)} \right)}{t^2}\\
&= \frac{\mathbb{E}[tZe^{tZ}] - \mathbb{E}[e^{tZ}] \log \mathbb{E}[e^{tZ}]}{t\varphi(t)} \\
&= \frac{Ent(e^{tZ})}{t^2 \varphi(t)}\\
\end{align*}

<<<<<<< Updated upstream
 To show the \autoref{Tensorization of Entropy}, we will need the following result. lemma::Given a random variable  $Y$ ,
=======

\end{proof}

To show the \autoref{Tensorization of Entropy}, we will need the following result.

\begin{lemma}[Duality of Entropy]
\label{Duality of Entropy}
Given a random variable  $Y$ ,
>>>>>>> Stashed changes


\begin{equation*}
Ent(Y) = \sup_{U: \mathbb{E} e^U \leq 1}\mathbb{E}[UY].
\end{equation*}

and furthermore, if we have a random variable  $U$  such that  $\mathbb{E}[UY] \leq Ent(Y)$  for any r.v.  $Y$ , then  $\mathbb{E}e^U\leq 1$ .

<<<<<<< Updated upstream
 proof::Consider
=======

\end{lemma}

proof::
\begin{proof}[Proof of~{\autoref{Duality of Entropy}}]
\label{proof:Duality of Entropy}
Consider
>>>>>>> Stashed changes


\begin{equation*}
Ent_{e^{U}P}[e^{-U}Y]
\end{equation*}

and notice we can compute that


\begin{equation*}
Ent_{e^{U}P}[e^{-U}Y] = Ent(Y) - \mathbb{E}[UY].
\end{equation*}

Then the proof follows from the fact that


\begin{equation*}
Ent_{e^{U}P}[e^{-U}Y] \geq 0
\end{equation*}

with equality when the random variable is constant, i.e. when  $e^{-U} = \frac{\mathbb{E}y}{Y}$  which yields a valid  $U$ , and so the inequality is attained for some  $U$ .

<<<<<<< Updated upstream
 proof::First we need lemma::Given a random variable  $Y$ ,


\begin{equation*}
Ent(Y) = \sup_{U: \mathbb{E} e^U \leq 1}\mathbb{E}[UY].
\end{equation*}

and furthermore, if we have a random variable  $U$  such that  $\mathbb{E}[UY] \leq Ent(Y)$  for any r.v.  $Y$ , then  $\mathbb{E}e^U\leq 1$ .

 Define  $W_i := \mathbb{E}[W| X_1, \dots, X_i]$ . Then
=======

\end{proof}

proof::
\begin{proof}[Proof of~{\autoref{Tensorization of Entropy}}]
\label{proof:Tensorization of Entropy}
First we need lemma::\autoref{Duality of Entropy} Define  $W_i := \mathbb{E}[W| X_1, \dots, X_i]$ . Then
>>>>>>> Stashed changes


\begin{align*}
Ent(W) &= \mathbb{E}\left[ W (\log W - \log \mathbb{E}W) \right]\\
& = \mathbb{E}\left[ W \sum_{i=1}^m (\log W_i - \log\mathbb{E}^{(i)} W_i) \right]\\
& = \mathbb{E} \sum_{i=1}^m \mathbb{E}^{(i)}[W (\log W_i - \log \mathbb{E}^{(i)} W_i)]\\
& \leq \mathbb{E} \sum_{i=1}^m Ent^{(i)}(W)
\end{align*}


<<<<<<< Updated upstream
proof::Suppose  $\bE Z = 0$  first, and let  $P = \mathrm{Law}(Z)$ . For  $t\in \bR$ , consider the exponentially tilted measure  $dP^{(t)} = \frac{e^{tZ}}{\bE e^{tZ}} dP$ . Then,
=======
\end{proof}

proof::
\begin{proof}[Proof of~{\autoref{Bounding the Entropy of a Sub-Gaussian Random Variable}}]
\label{proof:Bounding the Entropy of a Sub-Gaussian Random Variable}
Suppose  $\bE Z = 0$  first, and let  $P = \mathrm{Law}(Z)$ . For  $t\in \bR$ , consider the exponentially tilted measure  $dP^{(t)} = \frac{e^{tZ}}{\bE e^{tZ}} dP$ . Then,
>>>>>>> Stashed changes


\begin{align*}
\frac{Ent(e^{tz})}{\varphi(t)} &= \frac{\mathbb{E}[e^{tZ}tZ]}{\varphi(t)} - \frac{\varphi(t)\log \mathbb{E}e^{tZ}}{\varphi(t)} \\
& = \mathbb{E}_{P^{(t)}}[tZ] - \psi(t) \\
& = \mathbb{E}_{P^{(t)}} \log e^{tZ} - \psi(t) \\
& \leq \log \mathbb{E}_{P^{(t)}} [e^{tZ}] - \psi(t) \\
& = \log \mathbb{E} e^{2tZ} - 2 \psi(t) \\
& \leq \log \mathbb{E} e^{2tZ} \\
& \leq CK^2t^2
\end{align*}

This also holds for random variables of mean non-zero. Indeed,


\begin{equation*}
\frac{Ent(e^{tZ + C})}{\mathbb{E}[e^{tZ + C}]} = \frac{\mathbb{E}[e^C]}{\mathbb{E}[e^C]} \frac{Ent(e^{tZ})}{\varphi(t)} \leq CK^2t^2.
\end{equation*}

We were able to remove the negative term since  $\psi(t) \geq 0$  because one can notice that for t  $\geq 0$ ,


\begin{equation*}
-\psi(t) = -\log(\mathbb{E}e^{Zt}) \leq - \mathbb{E}[Zt] \leq -t \leq 0.
\end{equation*}


<<<<<<< Updated upstream

\section{Appendix}
\label{section:Appendix}
Here, we prove some basic properties of the entropy functional, and state its relationship with the Kullback-Leibler divergence. theorem::Let  $W\geq 0$  be any random variable with  $\bE W < \infty$  and  $\bE\, \phi(W)<\infty$ . 
=======
\end{proof}


\section{Appendix}
\label{section:Appendix}
Here, we prove some basic properties of the entropy functional, and state its relationship with the Kullback-Leibler divergence.

\begin{theorem}[Definition of entropy]
\label{Definition of entropy}
Let  $W\geq 0$  be any random variable with  $\bE W < \infty$  and  $\bE\, \phi(W)<\infty$ . 
>>>>>>> Stashed changes

\begin{itemize}
\item NOT IMPLEMENTED: Vector{Any}
\item NOT IMPLEMENTED: Vector{Any}
\item NOT IMPLEMENTED: Vector{Any}\end{itemize}
<<<<<<< Updated upstream
 proof::(1) follows directly from the convexity of  $\phi(w) = w \log w$  by Jensen's inequality. To prove (2), suppose  $W = c$  almost surely. Then  $\bE\phi(W) = \bE \phi(c) = \phi(c) = \phi(\bE W)$ , giving that  $Ent(W)=0$ . (3) follows from direct computation as
=======

\end{theorem}

proof::
\begin{proof}[Proof of~{\autoref{Definition of entropy}}]
\label{proof:Definition of entropy}
(1) follows directly from the convexity of  $\phi(w) = w \log w$  by Jensen's inequality. To prove (2), suppose  $W = c$  almost surely. Then  $\bE\phi(W) = \bE \phi(c) = \phi(c) = \phi(\bE W)$ , giving that  $Ent(W)=0$ . (3) follows from direct computation as
>>>>>>> Stashed changes


\begin{align*}
Ent(aW) &= \bE[aW \log aW] - \bE(aW)\log (\bE aW)\\
&= a\left\{ \bE[W\log W] + (\bE W)\log a - \bE(W)\log (\bE W) - (\bE W)\log a \right\} \\
&= a\,Ent(W).
\end{align*}

<<<<<<< Updated upstream
 The entropy functional is related to the usual Kullback-Leibler divergence of appropriately constructed measures. Suppose  $P = \mathrm{Law}(X_1, \dots, X_m)$  and let  $W^{(\lambda)} = e^{\lambda g(X_1, \dots, X_m)}$  for  $\lambda \in \bR$ . 
\section{Statement}
\label{section:Statement}
Define the corresponding  \textit{tilted}  measure defined by the density

\begin{equation*}
\frac{dP^{(\lambda)}}{dP} = \frac{e^{\lambda g(X_1, \dots, X_m)}}{\bE\,e^{\lambda g(X_1, \dots, X_m)}}.
\end{equation*}
 Then,


\begin{align*}
D_{\mathrm{KL}}(P^{(\lambda)}||P) &= \bE\left[\frac{dP^{(\lambda)}}{dP} \log \frac{dP^{(\lambda)}}{dP}\right] \\
&= \bE\left[\frac{dP^{(\lambda)}}{dP} \log \frac{dP^{(\lambda)}}{dP}\right] - \bE\left[\frac{dP^{(\lambda)}}{dP}\right]\log \bE \left[\frac{dP^{(\lambda)}}{dP}\right]\\
&= Ent\left(\frac{dP^{(\lambda)}}{dP}\right)\\
&= \frac{1}{\bE\,e^{\lambda g(X_1, \dots, X_m)}}\, Ent\left(e^{\lambda g(X_1, \dots, X_m)}\right).
\end{align*}

where the second equality is due to the fact that  $\bE  \left[ \frac{dP^{(\lambda)}}{dP} \right] = 1$ , and the last inequality follows from the positive homogeneity of entropy.


=======

\end{proof}

\autoref{Definition of entropy}
>>>>>>> Stashed changes

\printbibliography
\end{document}
