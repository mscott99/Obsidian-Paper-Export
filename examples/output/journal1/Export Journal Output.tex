\documentclass{article}
\usepackage{hyperref}
\usepackage{amsmath}
\usepackage{amsthm}
\usepackage{amssymb}
\usepackage{biblatex}
\addbibresource{bibliography.bib}
\newtheorem{theorem}{Theorem}
\newtheorem{lemma}[theorem]{Lemma}
\newtheorem{corollary}[theorem]{Corollary}
\newtheorem{definition}{Definition}
\title{Title}
\author{Author}
\begin{document}
\maketitle
\begin{abstract}
Hello blah bldsjlfdsjk sjdklfs

\end{abstract}

\section{Introduction}\label{section:Introduction}
More text
\section{Main Results}\label{section:Main Results}
We start with a definition:
\begin{definition}
\label{(k,d,n)-Generative Network with Piecewise Affine Activations}
Define a (k,d,n)-generative net with layer widths  $k_{0}, k_1, \dots, k_{d}$  with  $k = k_0$  and  $k_d = n$  and coefficients  $\{ l_i \}_{i \in [d]}\subseteq \mathbb{N}^d$  to be a function  $G:\field^k \to \field^n$  of the form
$
\begin{align*}
G(z) = \sigma_d \left( W^{(d)} \sigma_{d-1}\left( \cdots W^{(2)} \sigma_1 \left( W^{(1)} z \right) \right)\right), 
\end{align*}$
where for each  $i \in [d]$ ,   $\sigma_i:\mathbb{R}^{k_i}\to\mathbb{R}^{k_i}$  is a piecewise affine function with pieces generated by no more than  $l_i k_i$  hyperplanes.

\end{definition}
Our main theorem is the following:
\begin{theorem}
\label{Self-Difference of Affine Neural Net is Contained in Conical Piecewise Linear Function with Counted Pieces}
The self-difference of the range of a \autoref{(k,d,n)-Generative Network with Piecewise Affine Activations}  $G$  is contained in the range of a positive-homogeneous piecewise linear function  $\tilde{G}:\field^{2k+1}\to \field^{n}$  with a number of linear conical pieces  $M$  such that 

\log M \leq 2kd\log\left( \frac{e s}{k} \right),  
where

s = \left( \prod \textit{{i=1}^d \max(k, k} i)l_i \right)^{1/d}.  

\end{theorem}

\section{Proofs}\label{section:Proofs}
We now prove the main result:
\begin{proof}
Notice that  $G$  is a piecewise affine function with  $N$  pieces and \autoref{Counting the Number of Affine Components of a Neural Net with Affine Activations} lemma::
\begin{lemma}
A \autoref{(k,d,n)-Generative Network with Piecewise Affine Activations} will have at most

\log N \leq kd \log\left(\frac{e s}{k}\right)  
affine pieces where

s := \left( \prod \textit{{i=1}^d \max(k, k} i)l_i \right)^{1/d}  
is a geometric mean of the weighted widths of the layers of the network.

\end{lemma}

It follows that  $(x,y) \to G(x) - G(y)$  is also piecewise affine and has no more than  $2N$  pieces.
Then from  \autoref{Piecewise Affine Function is Contained in Conical Piecewise Linear Function with Extended Domain} lemma::
\begin{lemma}
Let  $f:\mathbb{R}^k\to\mathbb{R}^n$  be piecewise affine with  $N$  pieces. Then there exists a function  $\tilde{f}:\mathbb{R}^{k+1} \to \mathbb{R}^n$  which is piecewise linear with conical pieces such that  $\range(\tilde{f})\supseteq\range(f)$ . Furthermore,  $\tilde{f}((x,1)) \equiv f(x)$  and in the case that  $f$  has polyhedral pieces,  $\tilde{f}$  will have pieces that are polyhedral cones.

\end{lemma}
 there exists a function  $\tilde{G}:\mathbb{R}^{2k + 1} \to \mathbb{R}^n$ , and one can check that it has all the properties in the statement.

\end{proof}
We used two lemmas in the previous proof; \autoref{Counting the Number of Affine Components of a Neural Net with Affine Activations}, \autoref{Piecewise Affine Function is Contained in Conical Piecewise Linear Function with Extended Domain}. These are based on \cite{vershyninHighDimensionalProbabilityIntroduction2018}.We now show \autoref{Counting the Number of Affine Components of a Neural Net with Affine Activations}.
\begin{proof}
From the definition of a \autoref{(k,d,n)-Generative Network with Piecewise Affine Activations},  the  $ith$  layer has an activation that is piecewise affine with no more than  $l_i k_i$  affine pieces. This  $ith$  layer will have a in its domain contained in a finite number of subspaces that are of dimension of no more than  $\min(k,k_i)$ . Therefore consider the  $ith$  activation acting on a  $\min(k,k_i)-$ dimensional subspace. By
\autoref{Counting the Number of Cell Crossings of a Subspace} lemma::
\begin{lemma}
Let  $U$  be a  $k-$ dimensional hyperplane in general position in  $\field^n$ . Let there be  $P$  hyperplanes  $\{ H_i \}_{i \in [P]}$  in  $\field^n$ . These hyperplanes induce connected components between the hyperplanes that we call "cells". The subspace  $U$  may intersect at most

\sum_{i=0}^k {P \choose i} \leq \left( \frac{eP}{k} \right)^k    \ 
   cells.

\end{lemma}
 , the layer will generate no more than  $\left( \frac{el_i k_i}{\min(k, k_i)} \right)^{\min(k,k_i)}$   pieces for each subspace in its domain. Notice that this expression is monotonically increasing in  $k_i$ . Therefore, by increase  $k_i$  to  $\max(k,k_i)$  we get the upper bound

\left(  \frac{el \textit{i \max(k} i, k)}{k} \right)^{k}.  
It follows that the number of affine pieces  $N$  of the full network is such that

N \leq \prod \textit{{i=1}^d \left( \frac{el} i\max(k,k_i)}{k} \right)^k \leq \left(\frac{es}{k}\right)^{kd},  
where

s := \left( \prod \textit{{i=1}^d \max(k, k} i)l_i \right)^{1/d}.  

\end{proof}
And we now give the proof of lemma \autoref{Piecewise Affine Function is Contained in Conical Piecewise Linear Function with Extended Domain};
\begin{proof}
Letting  $S$  be the finite partition of  $dom(f)$  corresponding to the affine pieces of  $f$ , we have that  $f(x) = \sum_{s \in S} W_s x + b_s$  for some  $R^{n \times k}$  matrices  $\{ W_s \}$  and some vectors  $\{ b_s \}\subseteq \mathbb{R}^k$ . Let  $s_0 \in S$ . For any vector  $\tilde{x} \in \mathbb{R}^{k+1}$  with  $\tilde{x} = (\tilde{x}_{[k]}, \tilde{x}_{k+1})$ , let

\tilde{f}(\tilde{x}) := \indicator \textit{{\tilde{x}} k \leq 0} (W \textit{{s} 0} \tilde{x} \textit{[k]+ b} {s \textit{0}\tilde{x}} k) + \indicator \textit{{\tilde{x}} k > 0} \sum \textit{{s \in S} \indicator} {\frac{1}{\tilde{x} \textit{k}\tilde{x}} {[k]} \in s} (W \textit{s \tilde{x}} [k] + b \textit{s \tilde{x}} k).  
One can check that for this construction all the properties described in the statement hold.

\end{proof}
Created: \autoref{Temporal/Daily Notes/2023-03-27|2023-03-27}
\printbibliography
\end{document}
