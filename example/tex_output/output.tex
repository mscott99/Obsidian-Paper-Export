\documentclass{article}
\usepackage{amsmath}
\usepackage{amsthm}
\usepackage{aliascnt}
\usepackage{biblatex}
\usepackage{graphicx}
\usepackage{hyperref}

\theoremstyle{plain}
\newtheorem{theorem}{Theorem}[section]

% \newtheorem{prop}{Proposition}[section]
% \newtheorem{coro}{Corollary}
% \newtheorem{lem}{Lemma}

\renewcommand{\sectionautorefname}{Section} % name for \autoref
\renewcommand{\subsectionautorefname}{Section} % name for \autoref
\renewcommand{\subsubsectionautorefname}{Section} % name for \autoref

% prop
\newaliascnt{proposition}{theorem}% alias counter "<newTh>"
\newtheorem{proposition}[proposition]{Proposition}
\aliascntresetthe{proposition}
\providecommand*{\propositionautorefname}{Proposition} % name for \autoref

% coro
\newaliascnt{corollary}{theorem}% alias counter "<newTh>"
\newtheorem{corollary}[corollary]{Corollary}
\aliascntresetthe{corollary}
\providecommand*{\corollaryautorefname}{Corollary} % name for \autoref

% lem
\newaliascnt{lemma}{theorem}% alias counter "<newTh>"
\newtheorem{lemma}[lemma]{Lemma}
\aliascntresetthe{lemma}
\providecommand*{\lemmaautorefname}{Lemma} % name for \autoref


\theoremstyle{definition}
\newtheorem{definition}{Definition}[section]
\newtheorem{example}{Example}


\theoremstyle{remark}
\newtheorem{rmk}{Remark}[section]
\newtheorem{fact}[rmk]{Fact}
\newtheorem*{rmk*}{Remark}
\newtheorem*{fact*}{Fact}

\newenvironment{remark}
{\pushQED{\qed}\renewcommand{\qedsymbol}{$\diamond$}\rmk}
{\popQED\endrmk}

\providecommand*{\remarkautorefname}{Remark} % name for \autoref
\providecommand*{\rmkautorefname}{Remark} % name for \autoref
\providecommand*{\definitionautorefname}{Definition} % name for \autoref


\newcommand{\pr}{\mathbb{P}}
\DeclareMathOperator{\proj}{\Pi}
\DeclareMathOperator{\range}{\mathcal{R}}
\DeclareMathOperator{\domain}{dom}
\newcommand{\reals}{\mathbb{R}}
\newcommand{\nats}{\mathbb{N}}
\newcommand{\ints}{\mathbb{Z}}
\newcommand{\complex}{\mathbb{C}}
\newcommand{\kernel}{\text{ker}}
\newcommand{\bb}[1]{\mathbb{#1}}
\newcommand{\indicator}{\mathbb{1}}
\newcommand{\sphere}[1]{\mathbb{S}^{#1-1}}
\newcommand{\field}{\mathbb{C}}
\newcommand{\supp}{supp}
\newcommand{\rows}{rows}
\newcommand{\argmin}{argmin}
\newcommand{\support}{supp}
%\newcommand{\span}{span}
\newcommand{\bmat}[2]{\left[
        \begin{array}{#1}
            #2
        \end{array}
        \right]}

\addbibresource{bibliography.bib}

\title{Unevenly Fourier}
% \author{%
%   Aaron Berk\\
%   Department of Mathematics and Statistics\\
%   McGill University\\
%   Montr\'eal, QC, Canada\\
%   \texttt{aaron.berk@mcgill.ca}
%   \And
%   Simone Brugiapaglia\\
%   Department of Mathematics and Statistics\\
%   Concordia University\\
%   Montr\'eal, QC, Canada\\
%   \texttt{simone.brugiapaglia@concordia.ca}
%   \And
%   % Yaniv Plan% \\
%   % % \texttt{yaniv@math.ubc.ca}
%   % \And
%   % Matthew Scott\thanks{Corresponding author}% \\
%   % % \texttt{matthewscott@math.ubc.ca}
%   % \And
%   % Summer Sheng% \\
%   % % \texttt{xsheng@math.ubc.ca}
%   % \And
%   % \"{O}zg\"{u}r Y\i{}lmaz % \\
%   % % \texttt{oyilmaz@math.ubc.ca}
%   % \And
%   % % \\[-2.5em]
%   Yaniv Plan $\qquad$ Matthew Scott\thanks{Corresponding author} $\qquad$ Xia Sheng $\qquad$ \"{O}zg\"{u}r Y\i{}lmaz\\
%   Department of Mathematics\\
%   University of British Columbia\\
%   Vancouver, BC, Canada\\
%   \texttt{\{yaniv, matthewscott, xsheng, oyilmaz\}@math.ubc.ca}
% }

\begin{document}
\maketitle
\abstract{
  This is a simple example to showcase the Obsidian to LaTeX converter.

}
\section{Introduction}
\label{sec:introduction}
This document demonstrates the conversion of Markdown notes to a LaTeX document, including internal links and embedded content.
\begin{figure}[h]
\label{download.jpeg}
\centering
\includegraphics[width=0.5\textwidth]{Files/download.jpeg}
\caption{This is the caption}
\end{figure}
\begin{remark}
\label{main_note-statement}
I remark that this plugin is awsome.

\end{remark}

\section{Results}
\label{sec:results}

Given suitable definitions, the following follows.
\begin{align*}
\|\hat x - x_0\|_2
\leq \|x^\perp\|_2 + 3\|\tilde{F}x^\perp\|_2 + 3 \|\eta\|_2 + \frac{3}{2}\hat\varepsilon.
\end{align*}
where $\sum a_i = 0$. See~\hyperlink{lemma_1-proof}{the proof}.
$$\sum_{i=1}^k A_i$$
We also present the following lemma:
\begin{lemma}[lemma_1]
\label{lemma_1-statement}
Every even integer greater than 2 can be expressed as the sum of two prime numbers.
\end{lemma}

\begin{proof}[\hypertarget{lemma_1-proof}Proof of \autoref{lemma_1-statement}]
\label{lemma_1-proof}
This is Goldbach's Conjecture, which remains unproven as of the knowledge cutoff date (2021-09).
\end{proof}

The main theorem is:
\begin{theorem}[theorem_1]
\label{theorem_1-statement}
For every positive integer $n$, the sum of the first $n$ odd integers is equal to $n^2$.
\end{theorem}

\section{Proofs}
\label{sec:proofs}
Here is the proof for the main theorem. The proof is specifically for \autoref{theorem_1-statement}. I may or may not follow from \cite{rudelsonSparseReconstructionFourier2008}.
We proceed by induction.
Base case ($n = 1$): The sum of the first odd integer (1) is equal to $1^2$, which is true.
Inductive step: Assume that the sum of the first $k$ odd integers is equal to $k^2$. We want to show that the sum of the first $k+1$ odd integers is equal to $(k+1)^2$.
The sum of the first $k$ odd integers is $k^2$. The next odd integer is $(2k+1)$. Therefore, the

\printbibliography
\end{document}
